\begin{abstract}
  But, this brings us back to a relevant question: "Okay. I believe in the potential of the cryptocurrency ecosystem." But why specifically believe in Bitcoin's potential? There are better technologies coming out there, such as Hathor, Polkadot and Cardano. This is a crucial point for building our understanding. The main creator of Ethereum, which is now undergoing a migration from the Proof of Work protocol to the Proof of Stake, rightly cited the fact that Bitcoin was not adapting and changing and evolving so fast, as if saying it was falling behind. And what Vitalik (creator of Ethereum) puts as a deficiency, a problem, I see as something positive. Here is the reasoning in which I explain the reason.

  It's undeniable that Bitcoin has many shortcomings, however note, it doesn't need to evolve to address all of these shortcomings. The entire cryptocurrency ecosystem exists for just that. Despite the fact that change, evolution is relevant for a variety of reasons, Bitcoin's non-change doesn't just have a downside. There is also a non-change positive side. The ecosystem MUST evolve, but it also needs a solid foundation, and that requires perpetuity. And that implies slower changes, or perhaps almost non-changes.

  Of course, small changes, especially related to security, are relevant. But radical changes, such as block size, speed, protocol, etc. All this generates excessive dynamism. Better to leave such dynamism to other cryptocurrencies. Bitcoin was the first crypto to be created. For 10 years without operating failures, your network is synonymous with robustness. With 10 years of existence, there has been time for its source code to be seen and reviewed by scientists and engineers. There was already time for bugs to be fixed. There was time for people to digest and understand the advantages, disadvantages and limitations. There's already been enough time and enough hype to create enough marketing and make Bitcoin known enough.

  This all brings confidence. And that trust produces the value Bitcoin has. The entire existing ecosystem has already been built, intertwining in some way with the existence of Bitcoin. All this has value. Therefore, Bitcoin is the symbol representing the market to which it belongs. In that sense, an analogy with Swiss banks is reasonable. Swiss banks now charge negative interest to their large customers. Without a doubt, there are countries with more favorable legislation, positive interest rates, and several other attractions to attract large customers. And in effect, they will be able to capture customers. But still, there are a number of clients willing to leave their capital in Swiss banks even at negative interest. Yes, interest rates are the worst in the world, yes the legislation is not as favorable as possible, but... it's Switzerland. I trust Switzerland. They've been doing this for decades. It's stable. It's safe. It's trustable. I know it will go in year, it will go out year, and they won't suddenly change the legislation. I trust that no change in the political landscape will put my capital at risk.

  And so is Bitcoin for the cryptocurrency market. He doesn't need to, more than that, he MUST NOT CHANGE. The ecosystem needs to evolve, but it needs to stay as it is. What's more: Bitcoin is just the ideal crypto for "non-change", because the technology's inventor, known by the pseudonym Satoshi Nakamoto is anonymous, and there is no organization strong enough to "dict the course of the token", as the case of other projects.

\end{abstract}
