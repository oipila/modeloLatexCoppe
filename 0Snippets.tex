%
%
%FRAGMENTOS DE CÓDIGO PASSÍVEIS DE SEREM USADOS:
%
%
%

%Criando modo de inserir lista de graficos
\usepackage{float}
\usepackage{caption}
\newfloat{grafico}{htpb}{plt}[chapter]
\floatname{grafico}{Gráfico}
\newcommand{\listofgraficosname}{Lista de Gráficos}
\newcommand{\listofgraficos}{%
  \addcontentsline{toc}{chapter}{\listofgraficosname}
  \listof{grafico}{\listofgraficosname}
}

%Criando modo de inserir lista de quadros
\newfloat{quadro}{htpb}{plt}[chapter]
\floatname{quadro}{Quadro}
\newcommand{\listofquadrosname}{Lista de Quadros}
\newcommand{\listofquadros}{%
  \addcontentsline{toc}{chapter}{\listofquadrosname}
  \listof{quadro}{\listofquadrosname}
}

\listofquadros
\listofgraficos

\symbl{Symbol}{Symbol Definition}, and
for abbreviations \abbrev{Abbreviation}{Abbreviation Definition}

%Sequência para compilação de forma a não inserir listas na ToC:
%Compilar uma vez dentro do Atom mesmo, DEPOIS:
makeindex -s coppe.ist -o 0Principal.lab 0Principal.abx
makeindex -s coppe.ist -o 0Principal.los 0Principal.syx
%Depois, compilar de novo dentro do Atom mesmo.

%Esse só precisa usar se for compilar tudo por fora do Atom:
dvipdf 0Principal.dvi

%Capítulos iniciando no início da folha
\makeatletter
\renewcommand{\@makechapterhead}[1]{%
  {\noindent\raggedright\normalfont% Alignment and font reset
   \huge\bfseries \@chapapp\space\thechapter~~#1\par\nobreak}% Formatting
  \vspace{\baselineskip}% ...just a little space
}
\makeatother

%Alternativa descartada para criar epígrafes
\usepackage{dirtytalk}
\say{\textit{Beginning at the beginning,}} \textit{the King said gravely,}
\say{\textit{and go on till you come to the end; then stop.}} --- \textit{Lewis Carroll, Alice in Wonderland}

%Alternativa descartada para criar apêndices
\appendix
\chapter{Cientometria}

Um apêndice denominado "A".
\chapter{Bibliometria}

Um apêndice denominado "B".
